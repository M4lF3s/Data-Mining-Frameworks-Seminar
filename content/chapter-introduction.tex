% !TEX root = ../Seminararbeit-Data_Mining_Frameworks.tex
%


% =============================================================================
%
% Einleitung: Grundlagen zu Data Mining
%
% =============================================================================
\chapter{Data Mining Grundlagen}
\label{sec:intro}

\cleanchapterquote{Information is not knowledge.}{Albert Einstein}{(Theoretischer Physiker)}

%\Blindtext[2][2]
Der Begriff Data Mining bezeichnet zunächst einmal das Sammeln, Verarbeiten und
Analysieren von Daten und den damit verbundenen Informationsgewinn. Da
allerdings in der echten Welt eine große Bandbreite an Anwendungen und
Problemfeldern existiert, versteht man unter dem „minen von Daten“ ein sehr
weit gefächertes Feld an Methoden zur Datenverarbeitung. \\
\\
Data Mining hat in unserem Alltag längst Einzug gefunden, meistens bemerken wir
dies jedoch gar nicht. Nutzen wir beim Einkaufen beispielsweise eine
Bonuspunkte-Karte, sind in sozialen Medien aktiv oder stehen auf dem Weg zur
Arbeit im Stau generieren wir eine Unmenge an Daten. Diese werden von
Unternehmen gesammelt und anschließend ausgewertet. Innerhalb dieser Ansammlung
an Datensätzen finden sich Informationen über unsere Gewohnheiten, unsere
Interessen und über unser Verhalten. \\
Data Mining hilft uns eben jene Informationen interpretierbar zu machen und
somit besser zu verstehen wie Menschen mit ihrer Umwelt interagieren. \\
\\
Gleichzeitig muss man allerdings auch den Aspekt des Datenschutzes beachten.
Reicht das sammeln von Daten zu weit in die Privatsphäre eines einzelnen,
kann dies schnell zum Missbrauch dieser Informationen führen. \\
\\
Dass das Thema Data Mining kontrovers ist zeigt auch der Artikel „How
Companies Learn Your Secrets“ aus dem New York Times Magazine. \cite{NYT:12} Hier wollte
eine amerikanische Supermarktkette das Kaufverhalten ihrer Kunden untersuchen.
Um dies zu bewerkstelligen wurde den Kunden zunächst eine Identifikationsnummer
zugewiesen, sowie Namen, Kreditkarteninformationen und Email-Adresse
gespeichert. Unter Einbezug weiterer externer Datenquellen zur Demografie
konnten einige interessante Beobachtungen gemacht werden. So konnte die
Supermarktkette beispielsweise die Schwangerschaft von Frauen anhand der
Einkäufe erkennen und hat im Zuge dessen festgestellt, dass schwangere Frauen
im zweiten Trimester ihrer Schwangerschaft vermehrt geruchlose Lotionen kaufen.
Außerdem werden innerhalb der ersten 20 Wochen der Schwangerschaft häufiger
Zusatzstoffe wie Kalzium, Magnesium und Zink erworben. Nähert sich der Tag der
Entbindung werden zunehmend geruchlose Seife und extra große Wattepads in
Verbindung mit Desinfektionsmittel und Waschlappen gekauft. \\
Mit diesen Informationen war es möglich einen sog. „pregnancy prediction score“
zu errechnen welcher dazu genutzt wurde gezielt Werbung in Form von Gutscheinen
zu bestimmten Zeiten der Schwangerschaft zu verschicken. \\
Der Artikel beschreibt einen Vorfall bei dem ein wütender Mann in eine der
Filialen der Supermarktkette kam und sich beschwerte, dass seine Tochter
Gutscheine für Babykleidung und Krippen bekam. Der Mann wolle nicht, dass seine
Tochter von der Supermarktkette dazu ermutigt werde schwanger zu werden. Als
der Manager der Filiale später bei dem Vater anrief um sich zu entschuldigen
wurde ihm von diesem mitgeteilt, dass die Tochter tatsächlich schwanger sei, was der Vater
jedoch zum Zeitpunkt seiner Beschwerde nicht wusste. \\
\\
Dieses Beispiel zeigt, dass es wichtig ist präzise abzuwägen wie genau eine
Analyse der Daten sein sollte. \\
